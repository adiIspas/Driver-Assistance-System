\thispagestyle{plain}

\begin{center}
	\Large \textbf{Abstract}	
\end{center}

Lucrarea de față își propune abordarea unei teme de interes major din ultima perioadă de timp și anume implementarea de sisteme de asistență în trafic pentru mașini având o relevanță sporită și fiind o temă abordată tot mai des de producătorii auto. 

Principalele obiective ale lucrării sunt reprezentate de detectarea benzi de circulație curentă a mașinii, pentru aceasta fiind folosite diverse metode matematice de schimbare a perspectivei imaginii, IPM, și de detecție a liniilor definitorii benzi (sume cumulate).

Detecția de mașini, accentuat cu precădere pe detecția mașinii de pe banda curentă de circulație, reprezintă un alt obiectiv major al lucrării. Pentru aceasta fiind folosite noțiuni elementare de Machine Learning și Computer Vision precum vectori suport mașină, utilizați în antrenare descriptori HOG, utilizați pentru extragerea de caracteristici din fiecare frame al video-ului analizat.

Printre componentele secundare ale aplicației se numără estimarea distanței față de mașina detectată pe banda curentă de circulație și estimarea vitezei relative a acesteia raportată la mașina din care se inregistrează video-ul. Pentru aceaste componente au fost analizate date oferite de componenta de detecție a mașinii dintr-o succesiune de frame-uri.

În ceea ce privește rezultatele putem afirma că au fost obținute rezultate atât calitativ ca urmare a faptului că atât detectorul de mașini cât și cel de benzi oferă precizie de peste $90\%$, dar și cantitativ ca urmare a faptului că aplicația poate rula atât în condiții bune de iluminare, și anume ziua, dar și în condiții nu mereu prielnice de iluminare, și anume noaptea.
\vspace*{\fill}
