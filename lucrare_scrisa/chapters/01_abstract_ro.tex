\thispagestyle{plain}

\begin{center}
	\Large \textbf{Abstract}	
\end{center}

Prezenta lucrare de licență propune abordarea unei teme de interes major în ultima perioadă de timp și anume implementarea de sisteme de asistență în trafic pentru mașini. 

Prin noțiunea de sistem de asistență în trafic se înțeleg următoarele:

\begin{enumerate}
	\item Detectarea benzi curente de circulație a mașinii;
	\item Detectarea mașinii de pe banda curentă de circulație;
	\item Estimarea distanței față de mașina detectată pe banda curentă de circulație;
	\item Estimarea vitezei relative a mașinii de pe banda curentă de circulație față de mașina din care se intregistreaza video-ul.
\end{enumerate}

Lucrarea prezintă în capitolul de evaluare experimentală rezultate din punct de vedere cantitativ și calitativ. Arătăm că aplicația detectează mașinile și benzile cu o acuratețe foarte mare, in jur de $90\%$, putând rula în diferite condiții de iluminare.

Din prisma implementării, aplicația a fost dezvoltată în Matlab și rulată pe un sistem dotat cu procesor Intel i7 Quad Core, 2.60 GHz și 8 GB RAM. Biblioteca OpenCV a fost în Matlab și utilizată în diverse circumstanțe la care au fost adaugate și funcții din biblioteca VLFeat.

Din punct de vedere al vitezei de rulare a întregului proces, a fost obținut un timp mediu de rulare pe frame de 0.05 secunde cea ce înseamnă o rulare în timp real a intregii aplicații cu o capacitate de procesare de până la 30 de frame-uri pe secunde.  
\vspace*{\fill}